\documentclass[a4paper, 10pt]{article}
\usepackage[utf8]{inputenc}
\usepackage[spanish]{babel}
\usepackage{graphicx}
\usepackage{geometry}
\usepackage{listings}
\usepackage{amsmath}
\usepackage{amsfonts}
\usepackage{amssymb}
\usepackage{caratula}

\newcommand{\Z}{\mathbb{Z}}
\def\code#1{\texttt{#1}}
\newcommand\tab[1][0.5cm]{\hspace*{#1}}

\geometry{a4paper, margin=0.7in}

\begin{document}
    %Caratula
    \pagenumbering{gobble}
    \newpage

    \begin{center}
        \includegraphics{images/logo}
    \end{center}

    \materia{Organización de Computadoras}
    \submateria{Segundo Cuatrimestre 2017}
    \titulo{Trabajo Práctico 0}

    \integrante{Rodrigo De Rosa}{97799}{rodrigoderosa@outlook.com}
    \integrante{Marcos Schapira}{97934}{schapiramarcos@gmail.com}
    \integrante{Facundo Guerrero}{97981}{facundoiguerrero@gmail.com}
    \maketitle
    %Fin caratula
    %Table of contents
    \newpage
    \pagenumbering{roman}
    \tableofcontents
    %Fin table of contents
    %Informe
    \newpage
	\pagenumbering{arabic}
	\section{Diseño e Implementación}
		En este trabajo práctico inicial, cuyo objetivo es el de familiarizarnos con el
		entorno de desarrollo que utilizaremos en el cuatrimestre, se implementa un programa
		que recibe una entrada de texto e identifica los palíndromos que se encuentran en
		ella.
		\subsection{Estructura del problema}
			La entrada de texto previamente mencionada es una cadena de caracteres \emph{ASCII}
			sin ninguna restricción. Dentro de esta cadena son consideradas \emph{palabras} aquellas
			que están compuestas por los caracteres:
			\begin{itemize}
				\item $a-z$
				\item $0-9$
				\item $\_$ y $-$
			\end{itemize}
			\tab Cualquier otro caracter \emph{ASCII} es considerado un \emph{espacio}. Es decir,
			indica el fin de una \emph{palabra} y el comienzo de otra. Cabe destacar que una cadena
			con un sólo caracter es considerada \emph{palabra}.
		\subsection{Entorno}
			El trabajo se realizó en una máquina virtual \emph{NetBSD} (que simula tener un procesador
			\emph{MIPS}) montada por el emulador \emph{GXemul} en \emph{Ubuntu 17.04}.
		\subsection{Complicaciones}
			La principal complicación que surgió en el desarrollo del trabajo fue el hecho de que algunas
			librerías que existen en \emph{Ubuntu} no existen en \emph{NetBSD} (particularmente \emph{argp}),
			por lo que hubo que adaptarse a esta limitación y utilizar librerías que funcionaran en ambos. \\
			\tab Por último, un problema a resolver fue el de tener que enviar por \code{scp} todos los archivos
			que fueran modificados en \emph{Ubuntu} hacia \emph{NetBSD}. De todos modos este problema fue
			resuelto utilizando \code{sshfs} que permite utilizar la interfaz gráfica de \emph{Ubuntu} para
			modificar un directorio en la máquina virtual.
		\subsection{Desarrollo}
			El programa fue implementado en lenguaje C con sus librerías estándar y se utilizó la librería
			\code{getopt} para facilitar el parseo de los flags. \\
			\tab En cuanto al problema en sí, la solución implementada consiste en buscar las previamente
			llamadas \emph{palabras} dentro de una cadena de caracteres y verificar si invertidas son iguales
			a su versión original (esto indica que son palíndromos). Para hacer esto se utilizó una implementación
			de la función \code{strrev} que, si bien en las versiones más actuales de C viene en las librerías
			estándar, en \emph{NetBSD} no está dentro de estas.
			A continuación se detalla la documentación explicita de las funciones implementadas.
		\subsection{Manejo de errores}
			A lo largo del desarrollo del programa se definen ciertos errores para manejar posibles fallas del
			programa y así lograr un funcionamiento controlado y acorde. Estas son:
			\begin{itemize}

				\item \code{ALLOC$\_$ERROR}
				\\\textit{El error se puede dar al llamar a la función \code{malloc}.
				Junto a su mensaje específico se imprime a la vez el código generado por strerror en
				la anterior función.}
					\subitem \textbf{Mensaje:}
						\subsubitem \code{An error ocurred while allocating memory!}

				\item \code{REALLOC$\_$ERROR}
				\\\textit{El error se puede dar al llamar a la función \code{realloc}.
				Junto a su mensaje específico se imprime a la vez el código generado por strerror en
				la anterior función.}
					\subitem \textbf{Mensaje:}
						\subsubitem \code{An error ocurred while reallocating memory!}

				\item \code{INPUT$\_$OPEN$\_$ERROR}
				\\\textit{El error se puede dar al llamar la función \code{fopen}.
				Junto a su mensaje específico se imprime a la vez el código generado por strerror en
				la anterior función.}
					\subitem \textbf{Mensaje:}
						\subsubitem \code{An error ocurred while opening input file!}

				\item \code{OUTPUT$\_$OPEN$\_$ERROR}
				\\\textit{El error se puede dar al llamar la función \code{fopen}.
				Junto a su mensaje específico se imprime a la vez el código generado por strerror en
				la anterior función.}
					\subitem \textbf{Mensaje:}
						\subsubitem \code{An error ocurred while opening output file!}

				\item \code{RESULT$\_$WRITING$\_$ERROR}
				\\\textit{El error se puede dar al llamar la función \code{fprintf}
				si no se logró escribir todo el mensaje o si algo falló.
				Junto a su mensaje específico se imprime a la vez el código generado por strerror en
				la anterior función.}
					\subitem \textbf{Mensaje:}
						\subsubitem \code{An error ocurred while writing the result!}

				\item \code{PALINDROME$\_$ERROR$\_$MESSAGE}
				\\\textit{El error se puede dar al llamar a la función interna \code{get$\_$palindromes}.
				Esta devuelve \code{NULL} en caso de fallar (junto con su adecuado mensaje, explicado
				a continuación en el informe).}
					\subitem \textbf{Mensaje:}
						\subsubitem \code{An error ocurred while checking for palindromes!}

			\end{itemize}
		\subsubsection{Valores devueltos por la función \code{main}}
			Los siguientes códigos son mensajes devueltos por la función \code{main} al utilizar las funciones
			internas del programa(documentadas en la próxima sección del informe). Algunos de estos valores,
			en especial \code{FAIL} y \code{SUCCESS} son utilizados en otras funciones como valores booleanos
			False y True respectivamente.
			\begin{itemize}
				\item \code{SUCCESS} valor 0
					\\\textit{valor booleano de éxito.}
				\item \code{FAIL} valor 1
					\\\textit{valor booleano de falla.}
				\item \code{WRITING$\_$ERROR} valor 2
					\\\textit{ocurre cuando la función \code{write$\_$result} devuelve \code{FAIL}.}
				\item \code{PALINDROME$\_$ERROR} valor 3
					\\\textit{ocurre cuando la función \code{get$\_$palindromes} falla. Previamente se
					imprime el error \code{PALINDROME$\_$ERROR$\_$MESSAGE}}
				\item \code{BAD$\_$ARGUMENTS} valor 4
					\\\textit{ocurre cuando la función \code{process$\_$params} devuelve este mismo
					codigo al no poder procesar los parámetros correctamente.}
				\item \code{BAD$\_$INPUT$\_$PATH} valor 5
					\\\textit{ocurre cuando la función \code{open$\_$input} devuelve \code{FAIL}.}
				\item \code{BAD$\_$OUTPUT$\_$PATH} valor 6
					\\\textit{ocurre cuando la función \code{open$\_$output} devuelve \code{FAIL}.}
				\item \code{READING$\_$ERROR} valor 7
					\\\textit{ocurre cuando la función \code{read$\_$input} devuelve \code{FAIL o NULL}.}
			\end{itemize}
		\subsection{Documentación}
			Las siguientes funciones fueron implementadas con el objetivo de encontrar una solución al problema
			en cuestión:
			\begin{itemize}

				\item \code{char$*$ read$\_$input(FILE$*$ fp, size$\_$t size)}
				\\\textit{Lee el string entrante y lo devuelve.}
					\subitem \textbf{Parámetros:}
						\subsubitem \code{fp}: File Pointer de input
						\subsubitem \code{size}: Tamaño inicial del arreglo
					\subitem \textbf{Return:}
						\subsubitem El string leído o NULL en caso de fallar
						(imprimiendo el el error ocurrido)
					\subitem \textbf{Errores Posibles:}
						\subsubitem \code{REALLOC$\_$ERROR}
						\subsubitem \code{ALLOC$\_$ERROR}

				\item \code{FILE$*$ open$\_$input(char$*$ path)}
				\\\textit{Abre el input$\_$file y se devuelve su fp.
				Si el path es \code{NULL},
				se utiliza \code{DEFAULT$\_$INPUT} siendo en este caso stdin.}
					\subitem \textbf{Parámetros:}
						\subsubitem \code{path}: Dirección del archivo a abrir
					\subitem \textbf{Return:}
						\subsubitem File Pointer de input o \code{DEFAULT$\_$INPUT}
						en caso de no especificar un path.
					\subitem \textbf{Errores Posibles:}
						\subsubitem \code{INPUT$\_$OPEN$\_$ERROR}

				\item \code{FILE$*$ open$\_$output(char$*$ path)}
				\\\textit{Abre el output$\_$file y se devuelve su fp. Si el path es \code{NULL}
				se utiliza \code{DEFAULT$\_$OUTPUT} siendo en este caso stdout.}
					\subitem \textbf{Parámetros:}
						\subsubitem \code{path}: Dirección del archivo a abrir
					\subitem \textbf{Return:}
						\subsubitem File Pointer de output o \code{DEFAULT$\_$OUTPUT} en caso de
						no especificar un path.
					\subitem \textbf{Errores Posibles:}
						\subsubitem \code{OUTPUT$\_$OPEN$\_$ERROR}

				\item \code{int write$\_$result(FILE$*$ output$\_$fp, char$*$ result)}
				\\\textit{Escribe los palíndromos en el archivo indicado.}
					\subitem \textbf{Parámetros:}
						\subsubitem \code{output$\_$fp}: File Pointer del archivo a escribir
						\subsubitem \code{result}: String a escribir en el archivo
					\subitem \textbf{Return:}
						\subsubitem \code{SUCCESS o FAIL}
					\subitem \textbf{Errores Posibles:}
						\subsubitem \code{RESULT$\_$WRITING$\_$ERROR}

				\item \code{void close$\_$files(FILE$*$ fp1, FILE$*$ fp2)}
				\\\textit{Cierra los dos archivos recibidos.}
					\subitem \textbf{Parámetros:}
						\subsubitem \code{fp2}: File Pointer de archivo a cerrar
						\subsubitem \code{fp1}: File Pointer de archivo a cerrar

				\item \code{char$*$ strrev(char$*$ str)}
				\\\textit{Invierte la cadena recibida.}
					\subitem \textbf{Parámetros:}
						\subsubitem \code{str}: Cadena de caracteres a invertir
					\subitem \textbf{Return:}
						\subsubitem Cadena recibida, en orden inverso

				\item \code{char$*$ get$\_$palindromes(char$*$ string)}
				\\\textit{Devuelve un arreglo listo para escribir en un archivo
				que contiene en cada linea un palíndromo del string recibido.}
					\subitem \textbf{Parámetros:}
						\subsubitem \code{string}: Cadena a analizar en busca de palíndromos
					\subitem \textbf{Return:}
						\subsubitem Cadena que contiene solo palíndromos, con
						formato de uno por linea. En caso de error devuelve \code{NULL}
					\subitem \textbf{Errores Posibles:}
						\subsubitem \code{REALLOC$\_$ERROR}
						\subsubitem \code{ALLOC$\_$ERROR}

				\item \code{bool is$\_$palindrome(char$*$ string)}
				\\\textit{Verifica si una cadena es un palíndromo o no.}
					\subitem \textbf{Parámetros:}
						\subsubitem \code{string}: Cadena a analizar
					\subitem \textbf{Return:}
						\subsubitem Booleano
					\subitem \textbf{Errores Posibles:}
						\subsubitem \code{REALLOC$\_$ERROR}
						\subsubitem \code{ALLOC$\_$ERROR}

				\item \code{void print$\_$help()}
				\\\textit{Imprime por consola información de los comandos y sobre el
				uso del programa.}

				\item \code{void print$\_$version()}
				\\\textit{Imprime por consola la version del programa y los integrantes del grupo.}

				\item \code{int process$\_$params(int argc, char$**$ argv,
				char$**$ input$\_$file, char$**$ output$\_$file)}
				\\\textit{Procesa los parámetros de entrada del programa y almacena
				los paths correspondientes en los parámetros de la función.}
					\subitem \textbf{Parámetros:}
						\subsubitem \code{argc}: Cantidad de argumentos del programa
						\subsubitem \code{argv}: Vector de argumentos del programa
						\subsubitem \code{input$\_$file}: Puntero al string que contiene el path
						del input
						\subsubitem \code{output$\_$file}: Puntero al string que contiene el path
						del output
					\subitem \textbf{Return:}
						\subsubitem \code{SUCCESS o BAD$\_$ARGUMENTS} , en el segundo caso este valor
						es verificado y manejado en la función \code{main}.


			\end{itemize}
	\section{Ejecución}
		\subsection{Instrucciones para la compilación}
			Para compilar el programa se debe abrir una consola en el directorio donde se encuentra el archivo
			fuente (\code{tp.c}) y correr el comando: \code{gcc -Wall tp.c [-o OUTPUT]}.
		\subsection{Instrucciones para la ejecución}
			Suponiendo que nuestro archivo ejecutable fuera \code{tp0}, los comandos de consola para ejecutarlo
			son:
			\begin{itemize}
				\item \code{./tp0 -h} para ver la ayuda.
				\item \code{./tp0 -v} para ver la versión.
				\item \code{./tp0 -i ~/INPUT -o ~/OUTPUT}  para correr el programa con \code{INPUT} como archivo de entrada
				y \code{OUTPUT} como archivo de salida. Ambos son opcionales y son reemplazados por \code{stdin} y \code{stdout}
				respectivamente.
			\end{itemize}
		\subsection{Pruebas}
			Para probar el correcto funcionamiento del programa se utilizo un set de prueba. A continuación
			se muestra la composición y resultados de las ejecuciones de dicho set. Ademas, notar que
      desde la prueba 1 hasta la prueba 9, la entrada es mediante un archivo, es decir que
      se provee el archivo mediante \code{-i test$\_$inputN} ( con N = número de prueba), y la salida
      por terminal. Por último, vale aclarar que las pruebas se realizaron tanto en \code{Ubuntu}
      como en \code{NetBSD}.

      \subsubsection{Prueba 1}
        Caso de prueba provisto por la cátedra. Vale aclarar que un carácter es considerado
        palíndromo.\\

				\textbf{Entrada:}\\
				\tab\tab\code{Somos los primeros en completar el TP 0.}\\
				\tab\tab\code{Ojo que la fecha de entrega del TP0 es el martes 12 de Septiembre.}\\

        \textbf{Salida:}\\
				\tab\tab\code{Somos}\\
        \tab\tab\code{0}\\
				\tab\tab\code{Ojo}

			\subsubsection{Prueba 2}
        Este caso de prueba intenta demostrar el correcto funcionamiento de la
        detección de espacios. Como se puede ver, en la primera linea las palabras
        están separadas por el carácter espacio, pero en la segunda se linea se
        intenta demostrar que los caracteres no validos ( ver sección 1.1 )
        también funcionan como espacios. Ademas, vale notar que la palabra
        palíndroma se detecta sin importar mayúsculas o minúsculas.\\

				\textbf{Entrada:}\\
				\tab\tab\code{MeNEm neUquEn 1a2d323d2a1 adke}\\
				\tab\tab\code{pepe$)$nene$/$larral$=$dom-mod?a23$\_$32a}\\

        \textbf{Salida:}\\
				\tab\tab\code{MeNEm}\\
				\tab\tab\code{neUquEn}\\
				\tab\tab\code{1a2d323d2a1}\\
				\tab\tab\code{larral}\\
				\tab\tab\code{dom-mod}\\
				\tab\tab\code{a23$\_$32a}

			\subsubsection{Prueba 3}
        El objetivo de esta prueba es ver el funcionamiento de los caracteres
        validos que no sean letras ni números. Como se vio en la sección 1.1,
        \code{$\-$} y \code{$\_$} deben ser considerados como caracteres validos.
        Entonces esta prueba quiere demostrar el funcionamiento de dichos
        caracteres. Ademas, nuevamente notar que la detección de las palabras
        no es \code{sensitive}.\\

				\textbf{Entrada:}\\
        \tab\tab\code{aD$\-$2eT$\_$R$\_$Te2$\-$Da$/$4004$?$CheVr}\\
        \tab\tab\code{peep23$***$   avion${daad}$}\\
        \tab\tab\code{neUqUeN$\&$NarNran}\\

        \textbf{Salida:}\\
        \tab\tab\code{aD$\-$2eT$\_$R$\_$Te2$\-$Da}\\
        \tab\tab\code{4004}\\
        \tab\tab\code{daad}\\
        \tab\tab\code{neUqUeN}\\
        \tab\tab\code{NarNran}\\

      \subsubsection{Prueba 4}
        Esta prueba tiene el objetivo de corroborar el caso borde donde la
        entrada es vaciá.\\

        \textbf{Entrada:}\\
        \tab\tab\code{ }\\

        \textbf{Salida:}\\
        \tab\tab\code{ }\\

      \subsubsection{Prueba 5}
        El objetivo de esta prueba es verificar el correcto funcionamiento de
        la detección un único carácter.\\

				\textbf{Entrada:}\\
        \tab\tab\code{a}\\

        \textbf{Salida:}\\
        \tab\tab\code{a}\\


      \subsubsection{Prueba 6}
        El objetivo de esta prueba es corroborar el correcto funcionamiento del
        programa cuando se alcanza el limite inicial de tamaño del string que
        contiene las palabras palíndromas. En esta prueba, se ingresa una palabra
        palíndroma de 128 caracteres que es la capacidad máxima inicial.\\
        \textbf{La entrada para las siguientes pruebas es continua. Se hace un salto
        de linea con fines ilustrativos.}\\

        \textbf{Entrada:}\\
        \tab\tab\code{aaaaaaaaaaaaaaaaaaaaaaaaaaaaaaaaaaaaaaaaaaaaaaaaaaaaaaaaaaaaaaaa-\\-aaaaaaaaaaaaaaaaaaaaaaaaaaaaaaaaaaaaaaaaaaaaaaaaaaaaaaaaaaaaaaaa}\\

        \textbf{Salida:}\\
        \tab\tab\code{aaaaaaaaaaaaaaaaaaaaaaaaaaaaaaaaaaaaaaaaaaaaaaaaaaaaaaaaaaaaaaaa-\\-aaaaaaaaaaaaaaaaaaaaaaaaaaaaaaaaaaaaaaaaaaaaaaaaaaaaaaaaaaaaaaaa}\\

      \subsubsection{Prueba 7}
        Esta prueba intenta demostrar el funcionamiento adecuado del
        programa cuando se supera el tamaño inicial del string donde se guardan
        las palabras palíndromas. Aquí, se ingresa una palabra palíndroma de 129
        caracteres, es decir que se supera en 1 carácter la capacidad máxima.
        De esta manera, se esta forzando al programa a aumentar la capacidad
        máxima de almacenado. En otras palabras, se intenta probar el correcto
        funcionamiento del caso borde de la memoria, donde el programa debe
        hacer un realloc. Notar que se presentan dos casos de prueba, uno donde
        una misma palabra supera el limite, y en segundo lugar con palabras
        distintas.\\
        \textbf{La entrada para las siguientes pruebas es continua. Se hace un salto
        de linea con fines ilustrativos.}\\

        \textbf{Entrada 1:}\\
        \tab\tab\code{aaaaaaaaaaaaaaaaaaaaaaaaaaaaaaaaaaaaaaaaaaaaaaaaaaaaaaaaaaaaaaa-\\-aaaaaaaaaaaaaaaaaaaaaaaaaaaaaaaaaaaaaaaaaaaaaaaaaaaaaaaaaaaaaaaaaa}\\

        \textbf{Salida 1:}\\
        \tab\tab\code{aaaaaaaaaaaaaaaaaaaaaaaaaaaaaaaaaaaaaaaaaaaaaaaaaaaaaaaaaaaaaaa-\\-aaaaaaaaaaaaaaaaaaaaaaaaaaaaaaaaaaaaaaaaaaaaaaaaaaaaaaaaaaaaaaaaaa}\\

        \textbf{Entrada 2:}\\
        \tab\tab\code{aaaaaaaaaaaaaaaaaaaaaaaaaaaaaaaaaaaaaaaaaaaaaaaaaaaaaaaaaaaaaaa-\\-aaaaaaaaaaaaaaaaaaaaaaaaaaaaaaaaaaaaaaaaaaaaaaaaaaaaaaaaaaaa menem}\\

        \textbf{Salida 2:}\\
        \tab\tab\code{aaaaaaaaaaaaaaaaaaaaaaaaaaaaaaaaaaaaaaaaaaaaaaaaaaaaaaaaaaaaaaa-\\-aaaaaaaaaaaaaaaaaaaaaaaaaaaaaaaaaaaaaaaaaaaaaaaaaaaaaaaaaaaa}\\
        \tab\tab\code{menem}\\

      \subsubsection{Prueba 8}
        Esta prueba tiene como objetivo verificar el funcionamiento del programa
        cuando se alcanza el máximo inicial lectura. Entonces, se provee al
        programa una entrada de 256 caracteres.\\
        \textbf{La entrada para las siguientes pruebas es continua. Se hace un salto
        de linea con fines ilustrativos.}\\

        \textbf{Entrada:}\\
        \tab\tab\code{aaaaaaaaaaaaaaaaaaaaaaaaaaaaaaaaaaaaaaaaaaaaaaaaaaaaaaaaaaaaaaaa-\\-aaaaaaaaaaaaaaaaaaaaaaaaaaaaaaaaaaaaaaaaaaaaaaaaaaaaaaaaaaaaaaaa-\\-aaaaaaaaaaaaaaaaaaaaaaaaaaaaaaaaaaaaaaaaaaaaaaaaaaaaaaaaaaaaaaaa-\\-aaaaaaaaaaaaaaaaaaaaaaaaaaaaaaaaaaaaaaaaaaaaaaaaaaaaaaaaaaaaaaaa}\\

        \textbf{Salida:}\\
        \tab\tab\code{aaaaaaaaaaaaaaaaaaaaaaaaaaaaaaaaaaaaaaaaaaaaaaaaaaaaaaaaaaaaaaaa-\\-aaaaaaaaaaaaaaaaaaaaaaaaaaaaaaaaaaaaaaaaaaaaaaaaaaaaaaaaaaaaaaaa-\\-aaaaaaaaaaaaaaaaaaaaaaaaaaaaaaaaaaaaaaaaaaaaaaaaaaaaaaaaaaaaaaaa-\\-aaaaaaaaaaaaaaaaaaaaaaaaaaaaaaaaaaaaaaaaaaaaaaaaaaaaaaaaaaaaaaaa}\\

      \subsubsection{Prueba 9}
        El objetivo de esta prueba es corroborar el correcto funcionamiento del
        programa cuando se supera el tamaño inicial de lectura. Con dicho motivo,
        se ingresa una entrada de 257 caracteres, superando en 1 carácter la
        capacidad máxima. Al igual que la prueba 7, se esta probando el realloc
        pero en este caso para la lectura.
        \textbf{La entrada para las siguientes pruebas es continua. Se hace un salto
        de linea con fines ilustrativos.}\\

        \textbf{Entrada:}\\
        \tab\tab\code{aaaaaaaaaaaaaaaaaaaaaaaaaaaaaaaaaaaaaaaaaaaaaaaaaaaaaaaaaaaaaaaa-\\-aaaaaaaaaaaaaaaaaaaaaaaaaaaaaaaaaaaaaaaaaaaaaaaaaaaaaaaaaaaaaaaa-\\-aaaaaaaaaaaaaaaaaaaaaaaaaaaaaaaaaaaaaaaaaaaaaaaaaaaaaaaaaaaaaaaa-\\-aaaaaaaaaaaaaaaaaaaaaaaaaaaaaaaaaaaaaaaaaaaaaaaaaaaaaaaaaaaaaaaaa}\\

        \textbf{Salida:}\\
        \tab\tab\code{aaaaaaaaaaaaaaaaaaaaaaaaaaaaaaaaaaaaaaaaaaaaaaaaaaaaaaaaaaaaaaaa-\\-aaaaaaaaaaaaaaaaaaaaaaaaaaaaaaaaaaaaaaaaaaaaaaaaaaaaaaaaaaaaaaaa-\\-aaaaaaaaaaaaaaaaaaaaaaaaaaaaaaaaaaaaaaaaaaaaaaaaaaaaaaaaaaaaaaaa-\\-aaaaaaaaaaaaaaaaaaaaaaaaaaaaaaaaaaaaaaaaaaaaaaaaaaaaaaaaaaaaaaaaa}\\

      \subsubsection{Prueba 10}
        Se repite la prueba 1, esta vez con el objetivo de probar la entrada-salida vía archivos.
        La entrada se pasa por parámetro como  \code{-i test$\_$input1} y se pasa el archivo de salida
         \code{-i test$\_$output10}. La salida mostrada, es lo que contiene el archivo de salida
        anteriormente mencionado.\\

				\textbf{Entrada:}\\
				\tab\tab\code{Somos los primeros en completar el TP 0.}\\
				\tab\tab\code{Ojo que la fecha de entrega del TP0 es el martes 12 de Septiembre.}\\

        \textbf{Salida:}\\
				\tab\tab\code{Somos}\\
        \tab\tab\code{0}\\
				\tab\tab\code{Ojo}

      \subsubsection{Prueba 11}
        Se repite la prueba 2, esta vez con el objetivo de probar la entrada-salida vía terminal.
        Se correrá el programa sin ningún parámetro, y se hará el ingreso por entrada estándar
        (indicando su finalización con \code{Ctrl+D}) y se obtendrá la salida de la misma forma.\\

        \textbf{Entrada:}\\
        \tab\tab\code{MeNEm neUquEn 1a2d323d2a1 adke}\\
        \tab\tab\code{pepe$)$nene$/$larral$=$dom-mod?a23$\_$32a}\\

        \textbf{Salida:}\\
        \tab\tab\code{MeNEm}\\
        \tab\tab\code{neUquEn}\\
        \tab\tab\code{1a2d323d2a1}\\
        \tab\tab\code{larral}\\
        \tab\tab\code{dom-mod}\\
        \tab\tab\code{a23$\_$32a}

      \subsubsection{Prueba 12}
        Se repite la prueba 3, con el objetivo de verificar la entrada estándar y salida vía archivos.
        La salida se pasa por parámetros como \code{-i test$\_$input12}. La salida mostrada
        a continuación, es lo que contiene el archivo de salida anteriormente mencionado.\\

        \textbf{Entrada:}\\
        \tab\tab\code{aD$\-$2eT$\_$R$\_$Te2$\-$Da$/$4004$?$CheVr}\\
        \tab\tab\code{peep23$***$   avion${daad}$}\\
        \tab\tab\code{neUqUeN$\&$NarNran}\\

        \textbf{Salida:}\\
        \tab\tab\code{aD$\-$2eT$\_$R$\_$Te2$\-$Da}\\
        \tab\tab\code{4004}\\
        \tab\tab\code{daad}\\
        \tab\tab\code{neUqUeN}\\
        \tab\tab\code{NarNran}\\

	\section{Conclusiones}
		La principal conclusión que obtuvimos a partir del desarrollo de este trabajo práctico es que, al trabajar en \emph{NetBSD}
		se debe ser más cuidadoso al programar que en el caso de, por ejemplo, \emph{Ubuntu}. Con esto nos referimos a ciertas
		situaciones particulares en las que \emph{Ubuntu} resuelve ciertos problemas de la programación que \emph{NetBSD} no resuelve
		y resulta en un error. \\
		\tab Por ejemplo, en la línea 291 de \code{tp.c} se agrega un fin de linea que no es necesario en \emph{Ubuntu}, pues lo
		'resuelve solo', pero sí lo es en \emph{NetBSD} pues en este último, el programa no encuentra en fin de linea e imprime
		caracteres de más. \\
		\tab Por esta misma razón, notamos que es recomendable trabajar constantemente utilizando la carpeta virtual de \code{sshfs}
		para poder compilar y ejecutar el programa en \emph{NetBSD} en lugar de hacerlo en \emph{Ubuntu}. De todas maneras, al tener
		que desarrollar en \emph{Assembler}, no queda otra opción que hacer esto recién mencionado.
\end{document}
